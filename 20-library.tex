%\chapter{Modeling library}
\chapter{Modeling library}

The works mentioned in previous chapter created their models for various cryptographic problems mostly by hand.
While the results obtained are interesting, they are hard to reproduce by others.
Also this approach does not help to solve similar problems (like using a different hash function instead), as a new model would have to be created from scratch.

To address these issues we provide an easy to use and reusable library for modeling SAT instances.
While the library can be used for modeling any problem we specifically focus on making modeling cryptographic problems as simple as possible.
In this chapter we will state our goals for this library, describe its design and inner functionality.
We will also show examples of its use.

%%%%%%%%%%%%%%%%%%%%%%%%%%%%%%%%%%%%%%%%%%%%%%%%%%%%%%%%%%%%%%%%%%%%%
\section{Goals}
The main goals of our library are as follows:

~\\
\textbf{Existing implementation reuse:}
In order to simplify the modeling of cryptographic primitives as much as possible we want to allow reuse of existing implementations.
Most commonly used primitives -- such as hash functions, block and stream ciphers and others -- have widely available implementations in all popular programming languages.

The library should therefore allow using these implementations with only minor changes.
In addition to saving time this also makes the modeling less error-prone as we can build upon a well tested implementation.

~\\
\textbf{Output abstraction:}
The library should take care of generating the output in proper format for some SAT solver.
With solvers that support advanced features, such as \emph{XOR} clauses, it should be possible to take advantage of them.

~\\
\textbf{Model parsing:}
After successfully solving the instance with a SAT solver we obtain a model in form of a satisfying variable assignment.
The library should be able to load this model and map the truth assignment back to variables defined by the user.
This makes it easy to extract for example the colliding messages out of the model.


%%%%%%%%%%%%%%%%%%%%%%%%%%%%%%%%%%%%%%%%%%%%%%%%%%%%%%%%%%%%%%%%%%%%%
\section{Our approach}
To achieve the goals stated in previous section we take advantage of a technique called \emph{operator overloading}.
This is a feature present in many programming languages.
For our library we have decided to use the \emph{Python} language which also supports it.
Python has the additional advantage that it is very popular and therefore has implementations of virtually all commonly used cryptographic primitives.

\subsection{Operator overloading}
\label{sec:operator-overloading}
As the name suggests, operator overloading allows us to overload (override) existing behavior of operators in some programming language.
While the feature is only \emph{syntactic sugar} (which means it does not allow us to do anything more than would be possible without it; it just simplifies the syntax) it not only greatly increases the readability of the code, but also allows us to reuse existing implementations as we will show.

The reason this feature is useful in our library is that we can change the type of some variables in existing code without having to change anything else.
For example, we can take an implementation of some hash function and change the type of all variables from the built-in integers to our new data type.
Without operator overloading this could would not be possible, since operators such as addition or bit shift would not be defined for our new type by the language.
With operator overloading however we can provide these definitions ourselves.

Our library provides a new data type \emph{BitVector} that supports all the operations as the built-in integer type.
Since Python uses \emph{dynamic typing} it is sufficient to change the types of the constants used by the cryptographic primitive implementation.
All other variables are a result of operations on these constants and will therefore have the proper type.

\subsection{Boolean circuit creation}
The difference between the built-in integer type and \emph{BitVector} is that while the integer variable only holds one given value, our type instead stores how its value can be obtained from other variables.

More specifically, each time an operation is applied to one or more operands (variables of type \emph{BitVector} or constants) the result is another instance of type \emph{BitVector} which stores these operands.  That means that the output of some cryptographic primitive is not a single value but instead a boolean circuit representation. The circuit will form a directed acyclic graph.
%TODO figure of boolean circuit here

\subsection{Instance generation}
Once we have the boolean circuit for a model we can then take this representation and output a SAT instance using the Tseitin transformation described in section \ref{sec:tseitin}.

The order of clauses in the output is irrelevant, however since the circuit forms a DAG we can process it in topological order.
For every node (representing an operator applied to one or more operands) we first assign numbers to all required variables.
This is because the \emph{DIMACS} input format (described in section \ref{sec:dimacs}) used by most SAT solvers uses integers to refer to variables.

In most cases we have one variable for every bit of the vector.
However, an addition node needs additional carry variables.
On the other hand, for a negation node we don't need to introduce additional variables, as we can simply use the variables already assigned to the operand with reversed polarity.
Similarly for cyclic bit shift we can reuse existing variables but with different ordering.

Once all variables have been assigned integer values we pass through all the nodes again.
This time we generate the clauses which model the operator behavior.
The number of clauses required depends on the operator and the bit width of the \emph{BitVector} node.

\subsection{Solving and working with solution}
After the model has been turned into an instance we can run a SAT solver on this generated list of clauses and wait for it to terminate.
If the instance is satisfiable the solver will find a satisfying truth assignment and output it as truth values for all the variables.
The library will load, parse and store this data for later use.

We can then easily query any of the \emph{BitVector} variables for its value.
The mapping from variables to integer labels will be used to find the appropriate assignment and to reconstruct the value of the node.

\section{Using the library}
We will demonstrate the use of our library on two examples.
The first one is very simple and shows the entire process from start to finish.
The second example shows a real-world use case -- modifying existing \emph{SHA-1} implementation to use our library for instance generation.
We compare this modified implementation to a hand-crafted one from a previous work.

\subsection{Simple example}
Suppose we wish to find solutions to the boolean equation $X = A \land (B \oplus C)$.
We begin with importing the modeling library and defining the input variables $A$,$B$ and $C$ as $1$-bit vectors:
%\begin{figure}[H]
\begin{minted}[linenos,
			   numbersep=5pt,
               frame=lines,
               framesep=2mm]{python}
from instance import *
A, B, C = BitVector(1), BitVector(1), BitVector(1)
\end{minted}

Next, we can write the expression in the same way as we normally would.
The resulting variable $X$ will also be of type \emph{BitVector} and will store the boolean circuit representing this expression:
\begin{minted}[linenos,
			   firstnumber=3,
			   numbersep=5pt,
               frame=lines,
               framesep=2mm]{python}
X = A & (B ^ C)
\end{minted}

Now we can create a new instance, generate it using the variables we are interested in and solve it using any SAT solver using the \emph{DIMACS} standard.
Afterwards the satisfying assignment is easily accessible through the variables.
In this case we will simply print it out:

\begin{minted}[linenos,
               firstnumber=4,
			   numbersep=5pt,
               frame=lines,
               framesep=2mm]{python}
instance = Instance()
instance.emit([X])
instance.solve(['minisat'])

print([q.getValuation(instance) for q in [A, B, C, X]])
\end{minted}

The output might look like \texttt{[[False], [False], [False], [False]]}, which is indeed a valid solution to our equation.
However, suppose we wish the result $X$ to be true.
We can add this additional constraint by setting

\begin{minted}[linenos,
               firstnumber=4,
			   numbersep=5pt,
               frame=lines,
               framesep=2mm]{python}
X.bits = [True]
\end{minted}

\noindent before generating and solving the instance.
Now we obtain the output \texttt{[[True], [False], [True], [True]]} which is another valid solution with the additional property that the value of $X$ is true.

The entire program is less than ten lines long and very straightforward.
We will use the same two concepts -- replacing input variables with \emph{BitVector} and adding additional constraints -- in the next example.

\subsection{SHA-1 example}
We begin with a standard \emph{SHA-1} implementation based on one available online \cite{alt2013sha1}.
We extended it to support reduced instances.
Here we show the most important parts of this code:

\begin{minted}[linenos,
			   numbersep=5pt,
               frame=lines,
               framesep=2mm]{python}
# Round functions and constants
fs = [lambda a, b, c, d, e: (b & c) | (~b & d),
      lambda a, b, c, d, e: b ^ c ^ d,
      lambda a, b, c, d, e: (b & c) | (b & d) | (c & d),
      lambda a, b, c, d, e: b ^ c ^ d]
K = [0x5A827999, 0x6ED9EBA1, 0x8F1BBCDC, 0xCA62C1D6]

def sha1(message, rounds = 80):
    # Initialization
    h0, h1, h2, h3, h4 = 0x67452301, ...,  0xC3D2E1F0
    # (omitted) Message padding
    for pos in range(0, len(message), 64):
        # (omitted) Prepare message chunk W
        A, B, C, D, E = h0, h1, h2, h3, h4
        for i in range(rounds):
            F = fs[i//20](A, B, C, D, E)
            k = K[i//20]

            T = (leftrotate(A, 5) + F + E + k + W[i]) & 0xFFFFFFFF
            A, B, C, D, E = T, A, leftrotate(B, 30), C, D
        h0 = (h0 + A) & 0xFFFFFFFF
        # ...        
        h4 = (h4 + E) & 0xFFFFFFFF
\end{minted}

To modify this code to use our library and produce a SAT instance we need to perform only a few small changes.
First of all, we need to convert all constants to \emph{BitVector} objects of the appropriate size:
\pagebreak
% Many spaces to overflow # outside of document...
\begin{minted}[linenos,
			   numbersep=5pt,
               frame=lines,
               texcomments,
               firstnumber=6,
               framesep=2mm]{python}
K = [intToVector(x) for x in [0x5A827999, ...]]                                 #\setcounter{FancyVerbLine}{9}
h0, h1, h2, h3, h4 = [intToVector(x) for x in [0x67452301, ...]]
\end{minted}

Next we replace the \texttt{leftrotate} function with the \emph{CyclicLeftShift} provided by our library\footnote{Since Python does not provide an operator for cyclic left shift we can't use operator overloading in this case.}.
Since addition on $32$-bit \emph{BitVector} objects is automatically done modulo $2^{32}$ we can remove the unnecessary \emph{and mask}:
\begin{minted}[linenos,
			   numbersep=5pt,
               frame=lines,
               firstnumber=15,
               framesep=2mm]{python}
    for i in range(rounds):
        F = fs[i//20](A, B, C, D, E)
        k = K[i//20]

        T = CyclicLeftShift(A, 5) + F + E + k + Mvec[i]
        A, B, C, D, E = T, A, CyclicLeftShift(B, 30), C, D
    h0, h1, h2, h3, h4 = h0+A, h1+B, h2+C, h3+D, h4+E
\end{minted}

As we can see the changes required are minimal and trivial to perform.
For brevity we have omitted details such as message padding -- the full source code can be found in the attachment \cite{papay2016code} in file \emph{samples/sha1\_instance\_test.py}.
It also includes additional constraints on the output bits to find (partial) preimages.
After obtaining a solution the message is extracted and hashed using a reference implementation to ensure its validity.

\subsection{Comparison to existing work}
In \cite{nossum2012sat} a custom program of about $800$ lines is used to generate instances for preimage attacks on \emph{SHA-1}.
Our implementation using the modeling library we created is about ten times shorter.
It also includes verification of the solution, unlike in Nossum's work where two additional programs are required to parse and verify the SAT solver output.

In addition to requiring much less code to be written our approach is also much more readable.
Compare the code for the innermost loop where one of the four \emph{SHA-1} round functions is computed, shown in figure \ref{fig:code-comp-sha1}.

\newsavebox{\mintedboxleft}
\newsavebox{\mintedboxright}
\begin{sidewaysfigure}
\begin{lrbox}{\mintedboxleft}
\begin{minipage}[c]{.4\textwidth}
\vspace{.5cm}%
\begin{minted}[linenos,
			   numbersep=5pt,
%               frame=lines,
               framesep=2mm]{c}
if (i >= 0 && i < 20) {
  for (unsigned int j = 0; j < 32; ++j) {
    clause(-f[j], -b[j], c[j]);
    clause(-f[j], b[j], d[j]);
    clause(-f[j], c[j], d[j]);

    clause(f[j], -b[j], -c[j]);
    clause(f[j], b[j], -d[j]);
    clause(f[j], -c[j], -d[j]);
  }
} else if (i >= 20 && i < 40) {
  xor3(f, b, c, d);
} else if (i >= 40 && i < 60) {
  for (unsigned int j = 0; j < 32; ++j) {
    clause(-f[j], b[j], c[j]);
    clause(-f[j], b[j], d[j]);
    clause(-f[j], c[j], d[j]);

    clause(f[j], -b[j], -c[j]);
    clause(f[j], -b[j], -d[j]);
    clause(f[j], -c[j], -d[j]);
  }
} else if (i >= 60 && i < 80) {
  xor3(f, b, c, d);
}
\end{minted}
\end{minipage}
\end{lrbox}
\begin{lrbox}{\mintedboxright}
\begin{minipage}[c]{.35\textwidth}
\vspace{.5cm}%
\begin{minted}[linenos,
			   numbersep=5pt,
%               frame=lines,
               framesep=2mm]{python}
if 0 <= i < 20:
  f = (b & c) | (~b & d)
elif 20 <= i < 40:
  f = b ^ c ^ d
elif 40 <= i < 60:
  f = (b & c) | (b & d) | (c & d)
else:
  f = b ^ c ^ d
\end{minted}
\end{minipage}
\end{lrbox}
\centering
\subtop[Instance generating tool from \cite{nossum2012sat}]{\usebox{\mintedboxleft}}
\hspace{2cm}
\subtop[Adapted version of our instance generating code]{\usebox{\mintedboxright}}
\caption{Comparison of code for computing the \emph{SHA-1} round functions.}
\label{fig:code-comp-sha1}
\end{sidewaysfigure}

For clarity and fairness of comparison we modified our version of the code slightly to avoid \emph{Python} specific features such as \emph{lambda functions} which make our code even shorter.
It is clear that our approach leads to much more readable code and is easier to write.

For example, for the \emph{choice} round function used in first $20$ rounds we simply use the definition provided in the standard (see section \ref{sec:hash-functions}).
On the other hand, a set of six clauses to encode this function had to be found and used in the referenced work.

As we will discuss in section \ref{sec:expression-encoding}, our approach does lead to a less efficient encoding (in terms of number of variable and clauses in the resulting instance).
However, as experiments in \ref{sec:expression-encoding-eval} demonstrate, this does not have any measurable effect on the solving time.
Moreover, we added expression optimization support to our library that can be used to reduce the instances and in the case of the \emph{choice} function does in fact lead to an even more efficient encoding using just four clauses.

\subsection{The N-Queens problem}
As we have stated previously, even though we have so far focused on hash functions, our library is generic enough to be used for other problems.
We demonstrate this with the following simple example -- solving the famous \emph{N-Queens} problem.

In this problem we have a chessboard of size $N\times N$ and the task is to place upon it $N$ queens such that no two of them threaten each other.

First of all we define the parameter $N$ and declare the board, which we will represent as $N$ rows, each one being an $N$-bit vector.
\emph{True} bits will mean a queen is placed there, \emph{false} bits will represent empty squares.

\begin{minted}[linenos,
			   firstnumber=7,
			   numbersep=5pt,
               frame=lines,
               framesep=2mm]{python}
N = 8
board = [BitVector(N) for _ in range(N)]
\end{minted}

Now we need to enforce the necessary constraints.
We begin with the rule that there must be at least one queen in every row (which follows easily from the problem statement).
We can model this in the following way: we take the binary \emph{or} of every cyclic rotation of the row.
If there is at least one true bit anywhere in the row, all bits of the result will be true, otherwise all will be false.
We can then force all the bits of the result to be true.

\begin{minted}[linenos,
			   firstnumber=12,
			   numbersep=5pt,
               frame=lines,
               framesep=2mm]{python}
for row in board:
    rot = FalseVector(N)
    for i in range(N):
        rot |= CyclicLeftShift(row, i)
    rot.bits = [True]*N
\end{minted}

Next we need to make sure there is at most one true bit in each row, which we do in a similar way.
Suppose there are two true bits in the row separated by $k$ false bits.
If we take the binary \emph{and} of the row and the cyclic left shift of the row by $k$ bits we will obtain a vector with one true bit.
Since we do not know the value of $k$ we will take the binary \emph{or} of all such vectors for all values of $0 < k < N$:

\begin{minted}[linenos,
			   firstnumber=20,
			   numbersep=5pt,
               frame=lines,
               framesep=2mm]{python}
for row in board:
    rot = FalseVector(N)
    for i in range(1, N):
        rot |= row & CyclicLeftShift(row, i)
    rot.bits = [False]*N
\end{minted}

We will do a similar trick to make sure there is at most one queen on every diagonal.
We omit the code here for brevity.

Lastly we need to make sure there is precisely one queen in every column.
However since we already have rules to enforce precisely one queen in every row, it is enough to ensure \emph{at least one} queen in every column:

\begin{minted}[linenos,
			   firstnumber=38,
			   numbersep=5pt,
               frame=lines,
               framesep=2mm]{python}
row_or = FalseVector(N)
for row in board:
    row_or |= row
row_or.bits = [True]*N
\end{minted}

Full source code for this sample is available in the attachment (see appendix \ref{apx:library}) in file \emph{samples/nqueens.py}.
The output of the program shows the solution as a board, with the symbol \texttt{\#} representing a queen:

\begin{Verbatim}
$ python3 test_nqueens.py
Number of variables:          5384    
Number of clauses:           12104     
CPU time : 0.008 s
SATISFIABLE

...#....
.#......
......#.
..#.....
.....#..
.......#
....#...
#.......
\end{Verbatim}
%$%

Even large instances, with values of $N$ around $50$, can be solved in just a few seconds using this program.

\section{Hash function toolkit}
In addition to the modeling library we also provide a simple command line interface for generating custom instances called \emph{hashtoolkit}.
It allows us to specify the hash function, number of rounds and input message length.
Additionally the output digest bits can individually be set to either true, false or \emph{r} -- the same as a randomly generated reference digest.

In this example we find a $10$-round $64$-bit preimage on \emph{SHA-1}, where the first eight bits are equal to a reference message and next eight bits are equal to the hexadecimal byte $0f$:

%\begin{minted}[frame=lines,
%               framesep=2mm]{shell-session}
\newcommand*{\tcol}[2]{\textcolor{#1}{#2}}
\begin{Verbatim}[commandchars=&\[\]]
$ python3 hashtoolkit.py -h sha1 -l 64 -r 10 \
  -o '&tcol[red][rrrrrrrr]&tcol[green][0000]&tcol[blue][1111]' -s 'minisat'
...
SATISFIABLE
Reference message: b'\x06)\xeaVqz\xe1\x8a'
Reference digest: &tcol[red][44]c79c8b97df9297a4f551b3d14ec9aafe296a32

Message length 64 bits
Message bytes: b'NQ\xdbd\x8a\x06\x98\xde' rounds:  10
Message digest:   &tcol[red][44]&tcol[green][0]&tcol[blue][f]7115cc0483f042885b32247fe5eab998a8b4
\end{Verbatim}
%\end{minted}
%$%

Collisions can also be found using this tool.
Supported hash functions are \emph{MD5}, \emph{SHA-1} and \emph{SHA-3-512}.
More details about the tool and usage instructions can be found in appendix \ref{apx:hashtoolkit}.

%\caption{a}
%\end{figure}
% - simple example
% - sha1 implementation
%   - compare to nossum
%     - code size/...
%     - generating time?
%     - solving time -> in experiments chapter?
%     - instance size (clauses/vars)