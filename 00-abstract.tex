\begin{abstract}
SAT solvers are an universal tool for finding solutions to boolean satisfiability problems.
In the past they have been used for cryptographic problems, such as finding preimages for hash functions or obtaining the key for stream ciphers.
However these solutions are not easily reusable or modifiable.

In our work we create a modeling library that allows simple creation of SAT instances.
We specifically focus on various cryptographic problems, however the library is generic enough that it can be used for other purposes as well.

Using this library we then create models for several cryptographic hash functions.
Various SAT solvers, optimizations and heuristics are evaluated on these models to compare their performance.

These include the use of the \emph{Espresso} logic minimizer to reduce the instance size, forcing custom variable branching order with help of modified SAT solvers and others.

\keywords{SAT, cryptography, hash functions, heuristics}
%TODO keywords
\end{abstract}
