\documentclass[myposter,portrait]{sciposter}

%% uzitocne package
\usepackage{multicol}
\usepackage{color}
\usepackage{graphicx}

%% znaky s diakritikou
%\usepackage[utf8]{inputenc}
%\usepackage[T1]{fontenc}
% \usepackage[slovak]{babel} % slovenske delenie slov

%% definicia farieb
\definecolor{mainCol}{rgb}{0.91,0.82,0.74} % farba pozadia posteru
\definecolor{sectionCol}{rgb}{0,0,0} % farba nadpisu
\definecolor{textCol}{rgb}{0.2,0,0} % farba hlavneho textu
\definecolor{BoxCol}{rgb}{1,1,0.8} % farba boxu okolo nadpisov

\def\mysection#1{
{\color{sectionCol}\section*{\sc\bfseries #1}}}

\usepackage{fontspec}
\usepackage{polyglossia}
\setdefaultlanguage{english}
\usepackage{csquotes}
\usepackage{microtype}

\begin{document}
\setlength{\logowidth}{20cm}
\setlength{\titlewidth}{\textwidth}
\addtolength{\titlewidth}{-\logowidth}
\rightlogo[0.9]{fmfilogo-farebne}
\useleftlogofalse

\color{textCol}

\title{Evaluation of SAT-based\\ preimage attack optimizations}
\author{Ladislav P\'apay\\
        Supervisor: Martin Stanek}
\institute{%
Katedra informatiky, 
FMFI UK, Mlynská Dolina, 842~48~Bratislava
}
\maketitle

\begin{multicols*}{3}

\mysection{Introduction}
SAT solvers are an universal tool for finding solutions to boolean satisfiability problems.
In the past they have been used for cryptographic problems, such as finding preimages for hash functions or obtaining the key for stream ciphers.
However these solutions are not easily reusable or modifiable.
~\\

In our work we created a modeling library that allows simple creation of SAT instances.
We specifically focus on various cryptographic problems, however the library is generic enough that it can be used for other purposes as well.
~\\

Using this library we create models for several cryptographic hash functions.
Various SAT solvers, optimizations and heuristics are evaluated on these models to compare their performance.
These include the use of the \emph{Espresso} logic minimizer to reduce the instance size, forcing custom variable branching order with help of modified SAT solvers and others.

\mysection{Library design}
Previous results obtained in this area required creation of custom programs for modeling the problem and generating the SAT instances.
While the results obtained are interesting, they are hard to reproduce by others.
Also this approach does not help to solve similar problems (like using a different hash function instead), as a new model would have to be created from scratch.
~\\

To address these issues we provide an easy to use and reusable library for modeling SAT instances.
While the library can be used for modeling any problem we specifically focus on making modeling cryptographic problems as simple as possible.
~\\

The main features of our library are as follows:

\begin{description}
\item[Existing implementation reuse]~\\
In order to simplify the modeling as much as possible we allow reuse of existing implementations with only minor changes.
In addition to saving time this also makes the modeling less error-prone as we can build upon a well tested implementation.

\item[Output abstraction]~\\
The library takes care of generating the output in proper format for some SAT solver.

\item[Model parsing]~\\
After successfully solving the instance with a SAT solver we obtain a model in form of a satisfying variable assignment.
The library can then load this model and map the truth assignment back to variables defined by the user.
This makes it easy to extract for example the colliding messages out of the model.
\end{description}
~\\

We also provide a command line program called \emph{HashToolkit} for easy generation of preimage attack instances.
It supports several hash functions, variable message length and customizable preimage bits.

~\\	
\mysection{Preimage attacks on hash functions}
\mysection{Using the library}

\mysection{Expression optimization}

\mysection{Branching order optimization}

\mysection{Conclusion}

\mysection{Library}
TODO link+qr

TODO selected references

%\mysection{Introduction}
%Here we show how easy it is to prepare a poster for ŠVK.
%There are some differences in preparing a poster compared to
%preparing a paper:
%
%\begin{itemize}
%\item use \emph{less text}, since people are not going to stand
%      in front of your poster forever and read all your text,
%\item use \emph{more figures}, because they quickly draw the
%      eye of the reader to the most important points on your poster,
%\item use \emph{simple structure} (no numbered theorems, subsections,
%      or numbered figures)
%\item cite onle \emph{the most important references}
%\end{itemize}
%
%\mysection{Sample text}
%
%Let $S=[s_{ij}]$ ($1\leq i,j\leq n$) be a $(0,1,-1)$-matrix
%of order $n$. Then $S$ is a {\em sign-nonsingular matrix}
%(SNS-matrix) provided that each real matrix with the same
%sign pattern as $S$ is nonsingular. 
%In this paper we consider the evaluation of integrals of the 
%following forms:
%\begin{equation}
%\int_a^b \left( \sum_i E_i B_{i,k,x}(t) \right)
%         \left( \sum_j F_j B_{j,l,y}(t) \right) dt,\label{problem}
%\end{equation}
%\begin{equation}
%\int_a^b f(t) \left( \sum_i E_i B_{i,k,x}(t) \right) dt,\label{problem2}
%\end{equation}
%where $B_{i,k,x}$ is the $i$th B-spline of order $k$ defined over the
%knots $x_i, x_{i+1}, \ldots, x_{i+k}$.
%
%\begin{enumerate}
%\item Use Gauss quadrature on each interval.
%\item Convert the integral to a linear combination of
%      integrals of products of B-splines and provide a recurrence for
%      integrating the product of a pair of B-splines.
%\item Convert the sums of B-splines to piecewise
%      B\'{e}zier format and integrate segment
%      by segment using the properties of the Bernstein polynomials.
%\item Express the product of a pair of B-splines as a linear combination
%      of B-splines.
%      Use this to reformulate the integrand as a linear combination
%      of B-splines, and integrate term by term.
%\item Integrate by parts.
%\end{enumerate}
%
%Of these five, only methods 1 and 5 are suitable for our purposes.
%
%%\includegraphics[width=\columnwidth]{allmonkeys}
%
%
%
%\mysection{Some displayed equations}
%     By introducing the product topology on  $R^{m \times m} \times
%R^{n \times n}$  with the induced inner product
%\begin{equation}
%\langle (A_{1},B_{1}), (A_{2},B_{2})\rangle := \langle A_{1},A_{2}\rangle 
%+ \langle B_{1},B_{2}\rangle,\label{eq2.10}
%\end{equation}
%we calculate the Fr\'{e}chet derivative of  $F$  as follows:
%\begin{eqnarray}
% F'(U,V)(H,K) &=& \langle R(U,V),H\Sigma V^{T} + U\Sigma K^{T}\nonumber\\
%             && - P(H\Sigma V^{T} + U\Sigma K^{T})\rangle \nonumber \\
%         &=& \langle R(U,V),H\Sigma V^{T} + U\Sigma K^{T}\rangle\nonumber \\
%&=& \langle R(U,V)V\Sigma^{T},H\rangle + \nonumber\\
%  &&    \langle \Sigma^{T}U^{T}R(U,V),K^{T}\rangle.    \label{eq2.11}
%\end{eqnarray}
%
%In the middle line of (\ref{eq2.11}) we have used the fact that the range of
%$R$ is always perpendicular to the range of $P$.  The gradient $\nabla F$  of
%$F$, therefore,  may be interpreted as the
%pair of matrices:
%\begin{eqnarray}
% \nabla F(U,V) &=& (R(U,V)V\Sigma^{T},R(U,V)^{T}U\Sigma )\nonumber\\
% && \in R^{m \times m} \times R^{n \times n}.   \label{eq2.12}
%\end{eqnarray}
%
%Thus, the vector field
%\begin{equation}
%\frac{d(U,V)}{dt} = -g(U,V) 	\label{eq2.15}
%\end{equation}
%defines a steepest descent flow on the manifold  ${\cal O} (m) \times
%{\cal O} (n)$ for the objective function  $F(U,V)$.
%
%\columnbreak 
%
%\mysection{Numerical experiments} 
%
%We conducted numerical experiments 
%in computing inexact Newton steps for discretizations of a  
%{\em modified Bratu problem}, given by  
%\begin{eqnarray} 
%{\displaystyle \Delta w + c e^w + d{ {\partial w}\over{\partial x} } } 
%&=&{\displaystyle f \quad {\rm in}\ D, }\nonumber\\[-1.5ex]
%\label{bratu} \\[-1.5ex]
%{\displaystyle w }&=&{\displaystyle 0 \quad {\rm on}\ \partial D , } \nonumber
%\end{eqnarray} 
%where $c$ and $d$ are constants. The actual Bratu problem has $d=0$ and  
%$f \equiv0$. It provides a simplified model of nonlinear diffusion  
%phenomena, e.g., in combustion and semiconductors, and has been 
%considered by Glowinski, Keller, and Rheinhardt \cite{GloKR85}, 
%as well as by a number of other investigators; see \cite{GloKR85} 
%and the references therein. See also problem 3 by Glowinski and  Keller  
%and problem 7 by Mittelmann in the collection of nonlinear model 
%problems assembled by Mor\'e \cite{More}. The modified problem  
%(\ref{bratu}) has been used as a test problem for inexact Newton 
%methods by Brown and Saad \cite{Brown-Saad1}.  
%
%\def\gmres{{GMRES}} 
%\def\gmresm{{\rm GMRES($m$)}} 
%
%In our experiments, we took $D = [0,1]\times[0,1]$, $f \equiv0$, 
%$c=d=10$, and discretized (\ref{bratu}) using the usual second-order 
%centered differences over a $100\times100$ mesh of equally 
%spaced points in $D$. In \gmres($m$), we took $m=10$ and used fast  
%Poisson right preconditioning as before.
%
%\bigskip 
%%\includegraphics[width=\columnwidth]{fig}
%%\caption{Graph of the function $\sin(x)/x$.} 
% 
%
%%% zoznam literatury
%%\bibliographystyle{apalike}
%%\bibliography{references}

\end{multicols*}
\end{document}

